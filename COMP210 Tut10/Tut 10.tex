\documentclass[handout]{beamer}
\usepackage{amssymb,latexsym,amssymb,amsmath,amsbsy,amsopn,amstext,upgreek}
\usepackage{color,multicol}
\usepackage{graphicx,wrapfig,fancybox,watermark,graphics}
\usepackage{picins}
%\usepackage{emp}
%\usepackage{pstricks,pst-plot}
\usepackage{pgf}
\usepackage{movie15}
\usepackage{hyperref}
\usepackage{pdfpages}
\usepackage{listings,bera}
\definecolor{keywords}{RGB}{255,0,90}
\definecolor{comments}{RGB}{60,179,113}
\lstset{language=C,
keywordstyle=\color{keywords},
commentstyle=\color{comments}\emph}
\hypersetup{
    pdfpagemode=FullScreen, % show in full screen?
}
\usepackage{algorithm}
\usepackage{algorithmic}
\renewcommand{\algorithmicrequire}{\textbf{Input:}}
\renewcommand{\algorithmicensure}{\textbf{Output:}}
% reference entry
\usepackage{bibentry, natbib}
% reference style
\bibliographystyle{IEEEtran} 
%reference lib
\nobibliography{refs}

\usepackage[
	%compress,
	minimal,
	%nonav,
	red,
	%gold,
	%numbers,
	%nologo,
	polyu
	]{beamerthemeHongKong}
\usefonttheme[professionalfonts]{serif}

\title[Tutorial 10]{Tutorial 10: Homework \\ The Rest of the Assignment 4}
\author[COMP210]{Qu Xiaofeng\texorpdfstring{, Teaching Asistant\\\tiny{quxiaofeng.at.polyu@gmail.com, PQ702}}{}}
\institute{COMP210\\Discrete Structure}
\date{\today}

\begin{document}

\frame{\titlepage}

\section*{Table of Contents}

    \begin{frame}{\secname}
        \tableofcontents
    \end{frame}

\AtBeginSubsection[] {
    \begin{frame}<handout:0>{Outline}
        \tableofcontents[current,currentsubsection]
    \end{frame}
}



\section{Problems}



    \subsection{Problem 2}
    
        \begin{frame}[c]{\subsecname}
            \emph{Write an algorithm that receives as input the matrix of a relation R and tests whether R is transitive.}\\$\;$\\\pause
            Let M be the n $\times$ n binary matrix representing the relation R,\\
            namely, $M_{ij}=1$ if iRj and vice versa\\
            R is transitive $\forall$i,j,k  if iRj, jRk then iRk\\
            The basic idea to use matrix to test transitive ralation is as follows:\\
            if M$\times$M contains no extra 1\rq{}s compared to M. Or M-M$\times$M has only zero or 1 entries.
        \end{frame}
        
        \begin{frame}[c,shrink]{\subsecname\ cont.}          
            \begin{algorithm}[H]
                \caption {Transitivity Test}
                \label{alg1}
                \begin{algorithmic}[1]
                    \REQUIRE Matrix M, Integer n
                    \FOR{i $\leftarrow$ 1 \TO n}
                    \FOR{j $\leftarrow$ 1 \TO n}
                    \STATE $A_{ij} \leftarrow$ 0
                    \FOR{k $\leftarrow$ 1 \TO n}
                    \STATE $A_{ij} \leftarrow M_{ik} \AND M_{kj}$
                    \ENDFOR
                    \IF{$A_{ij} > 0$ \AND$M_{ij}=0$}
                    \PRINT Not Transitive
                    \RETURN
                    \ENDIF
                    \ENDFOR
                    \ENDFOR
                    \PRINT Transitive \qed
                \end{algorithmic}
            \end{algorithm}        
        \end{frame}



    \subsection{Problem 3}
    
        \begin{frame}[c]{\subsecname}
            Write an algorithm that returns the index of the first occurrence of the value key in the sequence $s_1,\ldots,s_n$. If key is not in the sequence, the algorithm returns the value 0. Example: If the sequence is \\\centering{12 11 12 23} \\and key is 12, the algorithm returns the value 1. Please give the execution trace of your algorithm step-by-step for the input sequence 11 23 5 6, key = 4.
        \end{frame}
        
        \begin{frame}[c,shrink]{\subsecname\ cont.}          
            \begin{algorithm}[H]
                \caption {Find Key}
                \label{alg2}
                \begin{algorithmic}[1]
                    \REQUIRE Integer Array s, Integer key, Integer n
                    \FOR{i $\leftarrow$ 1 \TO n}
                    \IF{s(i) = key}
                    \RETURN i
                    \ENDIF
                    \ENDFOR
                    \RETURN 0 \qed
                \end{algorithmic}                
            \end{algorithm}        
        \end{frame}
        
        \begin{frame}[c,shrink]{\subsecname\ cont.}          
            \begin{algorithm}[H]
                \caption {Trace}
                \label{alg3}
                \begin{algorithmic}[1]
                    \REQUIRE Integer Array s = (11,23,5,6), Integer key = 4,\\ Integer n = 4
                    \STATE $i \leftarrow 1$
                    \STATE $s(i) = 11 \neq 4$
                    \STATE $i \leftarrow 2$
                    \STATE $s(i) = 23 \neq 4$
                    \STATE $i \leftarrow 3$
                    \STATE $s(i) = 5 \neq 4$
                    \STATE $i \leftarrow 4$
                    \STATE $s(i) = 6 \neq 4$
                    \RETURN 0 \qed
                \end{algorithmic}                
            \end{algorithm}        
        \end{frame}




    \subsection{Problem 4}
    
        \begin{frame}[c,shrink]{Common Growth Functions}
            \begin{table}[tp]%
                \caption{Common Growth Functions\ (Table 4.3.3)}
                \label{cgf}\centering%
                \begin{tabular}{ll}
                    \hline%\toprule%
                    Theta Form             & Name         \\\hline%\otoprule%
                    $\Theta(1)         $   & Constant     \\
                    $\Theta(\lg\lg{n}) $   & Log log      \\
                    $\Theta(\lg{n})    $   & Log          \\
                    $\Theta(n)         $   & Linear       \\
                    $\Theta(n\lg{n})   $   & n log n      \\
                    $\Theta(n_2)       $   & Quadratic    \\
                    $\Theta(n_3)       $   & Cubic        \\
                    $\Theta(n_k)       $   & Polynomial   \\
                    $\Theta(c_n)       $   & Exponential  \\
                    $\Theta(n!)        $   & Factorial    \\\hline%\bottomrule
                \end{tabular}
            \end{table}
        \end{frame}
    
        \begin{frame}[c]{\subsecname.2}
            Select a theta notation from Table 4.3.3 for: $2\lg{n}+4n+3n\lg{n}$\\$\;$\\\pause
            \[\lg n<n<3n\lg n\]
            $\Rightarrow$ The dominating term should be $3n\lg{n}$\\
            So\[f(n)=\Theta(n\lg{n})\] \qed
        \end{frame}
    
        \begin{frame}[c]{\subsecname.4}
            Select a theta notation from Table 4.3.3 for each express in the following:$\frac{(n^2+\lg{n})(n+1)}{n+n^2}$\\$\;$\\\pause
            For all $n > 0$, simplify\\
            \begin{align*}
            f(n)&=\frac{(n^2+\lg{n})(n+1)}{n+n^2}\\
            &=\frac{(n^2+\lg{n})(n+1)}{n(n+1)}\\
            &=\frac{n^2+\lg{n}}{n}=n+\frac{\lg{n}}{n}\\
            &=\Theta(n) &\qed
            \end{align*}
        \end{frame}
    
        \begin{frame}[c]{\subsecname.5}
            Select a theta notation from Table 4.3.3 for each express in the following:$2+4+8+\cdots+2^n$\\\pause
            \begin{align*}
            f(n)&=2^1+2^2+2^3+\cdots+2^n\\
            &=\sum_{i=1}^{n}2^i\\
            &=\frac{2-2^{n+1}}{1-2}\\
            &=2\times2^n-2\\
            &=\Theta(2^n) &\qed
            \end{align*}
        \end{frame}
    
        \begin{frame}[c]{\subsecname.6}
            Select a theta notation from Table 4.3.3 for each express in the following:$f(n)+g(n)$, where $f(n)=6n^3+2n^2+4$ and $g(n)=\Theta(n\lg{n})$.\\\pause
            \begin{align*}
            f(n)&=6n^3+2n^2+4\\
            &=\Theta(n^3)\\
            f(n)+g(n)&=\Theta(n^3)+\Theta(n\lg{n})\\
            &=\Theta(n^3) &\qed
            \end{align*}
        \end{frame}



    \subsection{Problem 5}
    
        \begin{frame}[fragile]{\subsecname.2}
            Express in theta notation the number of times the statement x = x + 1 is
executed.
            \begin{lstlisting}
j = n;
while( j >= 1 ) {
    for ( i = 1; i < j+1; i ++ ) {
        x = x + 1;
    }
    j = floor( j/3 );
}
            \end{lstlisting}
            where floor(x) = $\lfloor x\rfloor$.
\end{frame} 

    
        \begin{frame}[c]{\subsecname.2 cont.}
            \begin{align*}
            1\leq\frac{n}{3^k}\leq3&\Rightarrow3^k\leq n\leq3^{k+1}  \Rightarrow\frac{1}{3^k}\geq\frac{1}{n}\geq\frac{1}{3^{k+1}}\\
                                   &\Rightarrow \frac{1}{3n}\leq\frac{1}{3^{k+1}}\leq\frac{1}{n}
            \end{align*}
            \[f(n)=n+\frac{n}{3}+\ldots+\frac{n}{3^k}=n\cdot\frac{1-\frac{1}{3^{k+1}}}{1-\frac{1}{3}}\]
            \[n\cdot\frac{1-\frac{1}{n}}{\frac{2}{3}}\leq f(n)\leq n\cdot\frac{1-\frac{1}{3n}}{\frac{2}{3}}\]
            \[\frac{3n-3}{2}\leq f(n)\leq\frac{3n-1}{2}\]
            So $f(n)=\Theta(n)$ \qed
        \end{frame}
    
        \begin{frame}[fragile]{\subsecname.3}
            Express in theta notation the number of times the statement x = x + 1 is
executed.
            \begin{lstlisting}
i = 2;
while( i < n ) {
    i = i * i;
    x = x + 1;
}
            \end{lstlisting}
\end{frame}
    
        \begin{frame}[c]{\subsecname.3 cont.}
            \begin{center}
            \begin{tabular}{lll}
                    \hline%\toprule%
                    n               & i          & times   \\\hline%\otoprule%
                    $3=2+1      $   & 2          & 1       \\
                    $5=2^2+1    $   & 2,4        & 2       \\
                    $17=2^4+1   $   & 2,4,16     & 3       \\
                    $257=2^8+1  $   & 2,4,16,256 & 4       \\\hline%\bottomrule
            \end{tabular}\pause
            \end{center}
            \[{2^2}^k = n \Rightarrow k=lg{lg{n}}=\Theta(\lg{\lg{n}}) \qed\]
        \end{frame}



\section{Problems cont.}

    \subsection{Problem 7}
    
        \begin{frame}[c]{\subsecname}
            Show that if n is a power of 2, say $n=2^k$, then $\sum^k_{i=0}\lg(\frac{n}{2^i})=\Theta(\lg^2{n})$.\\\pause
            From the question, we get $k=\lg{n}$
            \begin{align*}
            &\sum_{i=0}^{k}\lg{\frac{n}{2^i}}\\
            =&\sum_{i=0}^{k}(\lg{n}-i\lg{2})=\sum_{i=0}^{k}\lg{n}-\sum_{i=0}^{k}i\\
            =&(\lg{n}+1)\lg{n}-\frac{k(k+1)}{2}\\
            =&\frac{\lg{n}(\lg{n}+1)}{2}\\
            =&\Theta(\lg^2{n}) &\qed
            \end{align*}
        \end{frame}



    \subsection{Problem 8}
    
        \begin{frame}[c]{\subsecname.1}
            Show, by consulting the figure, that $\frac{1}{2}+\frac{1}{3}+\ldots+\frac{1}{n}<\log_e{n}$\\
%\setlength{\unitlength}{1cm}
%\begin{picture}(6,4)(-1,-1)
%\put(-0.5,0){\vector(1,0){5}}
%\put(4.7,-0.1){$n$}
%\put(0,-0.5){\vector(0,1){3}}
%\put(-0.1,2.7){$f(n)$}
%\multiput(0,1)(0.2,0){6}{\line(1,0){0.15}}
%\multiput(0,0.5)(0.2,0){12}{\line(1,0){0.15}}
%\multiput(0,0.33)(0.2,0){18}{\line(1,0){0.15}}
%\multiput(1,0)(0,0.2){6}{\line(0,1){0.15}}
%\multiput(2,0)(0,0.2){3}{\line(0,1){0.15}}
%\multiput(3,0)(0,0.2){2}{\line(0,1){0.15}}
%\put(4,2){$y=\frac{1}{x}$}
%\qbezier(4,0.25)(0.5,0.5)(0.5,2)
%\put(-1,-1){\circle*{0.2}}
%\end{picture}
        \end{frame}
    
        \begin{frame}[c]{\subsecname.2}
            Show, by consulting the figure that $\log_e{n}<1+\frac{1}{2}+\ldots+\frac{1}{n}$
        \end{frame}
    
        \begin{frame}[c]{\subsecname.3}
            Use parts (a) and (b) to show that $1+\frac{1}{2}+\ldots+\frac{1}{n}=\Theta(\lg{n})$
        \end{frame}
    
        \begin{frame}[c]{\subsecname.4}
            A robot can take steps of 1 meter, 2 meters, or 3 meters. Write an algorithm to list all the ways that the robot can walk n meters.
        \end{frame}
        
        \begin{frame}[c,shrink]{\subsecname.4\ cont.}          
            \begin{algorithm}[H]
                \caption {robotwalk}
                \label{alg4}
                \begin{algorithmic}[1]
                    \REQUIRE Integer n, String s
                    \IF{$n = 1$}
                    \PRINT $s+''1 meter\ end''$
                    \RETURN
                    \ELSIF{$n = 2$}
                    \PRINT $s+''1 meter\ ''+''1 meter\ end''$
                    \PRINT $s+''2 meters\ end''$
                    \RETURN
                    \ELSIF{$n = 3$}
                    \PRINT $s+''1 meter\ ''+''1 meter\ ''+''1 meter\ end''$
                    \PRINT $s+''2 meters\ ''+''1 meter\ end''$
                    \PRINT $s+''1 meter\ ''+''2 meters\ end''$
                    \PRINT $s+''3 meters\ end''$
                    \RETURN
                    \ELSE
                    \STATE $s\prime = s + ''1 meter\ ''$
                    \STATE $robotwolk(n-1,\ s\prime)$
                    \STATE $s\prime = s + ''2 meters\ ''$
                    \STATE $robotwolk(n-2,\ s\prime)$
                    \STATE $s\prime = s + ''3 meters\ ''$
                    \STATE $robotwolk(n-3,\ s\prime)$
                    \ENDIF \qed
                \end{algorithmic}
            \end{algorithm}        
        \end{frame}



    \subsection{Problem 9}
    
        \begin{frame}[c]{\subsecname}
            How many different car license plates can be constructed if the licenses contain three letters followed by two digits if repetitions are allowed? if repetitions are not allowed.\\$\;$\\\pause
            the number of letters = 26\\
            the number of digits = 10\\\pause
            repetitions allowed $26^3\cdot10^2=1757600$\\
            repetitions not allowed $26\times25\times24\times10\times9=1404000$ \qed
        \end{frame}



    \subsection{Problem 11}
    
        \begin{frame}[c]{\subsecname}
            A committee composed or Morgan, Tyler, Max, and Leslie is to select a president
and secretary. How many selections are there in which Max is president or secretary.\\\pause
		 \begin{itemize}
		     \item
if the president and the secretary can not be the same person
                \begin{itemize}
                    \item For Max is a president, 3 selections.
                    \item For Max is a secretery, 3 selections.
                \end{itemize}
                6 selections in total.
                \item
if the president and the secretary can be the same person
                \begin{itemize}
                    \item For Max is a president, 3 selections.
                    \item For Max is a secretery, 3 selections.
                    \item For Max is both the president and the secretery, 1 selection.
                \end{itemize}
                7 selections in total. \qed
		\end{itemize}
        \end{frame}



\section{Problems cont.}

    \subsection{Problem 12}
    
        \begin{frame}[c]{\subsecname.1}
            For integers from 5 to 200, inclusive. How many are greater than 101 and do not contain the digit 6?\\$\;$\\\pause
            numbers 1xx do not contain digit 6: $9\times9$\\
            + ``200'' - ``100'' - ``101''\\
            81+1-2=80 \qed
        \end{frame}
    
        \begin{frame}[c]{\subsecname.3}
            For integers from 5 to 200, inclusive. How many have the digits in strictly increasing order? (Examples are 13, 147, 8)\\$\;$\\\pause          
                \begin{tabular}{lcr}
                     Single digit		& 5,6,7,8,9 	& 5 	\\
                     Double digits (xx)
                                & $ 1x\;\ x = 2 \rightarrow 9$ & \\
                                & $ 2x\;\ x = 3 \rightarrow 9$ &  	\\
                                & $ \Downarrow $ 	          &  	\\
                                & $ 8x\;\ x = 9 $ 	          & $ 8 + 7 + \cdots + 1 = 36 $ 	\\
                     Three digits (1xx)
                                & $ 12x\;\ x = 3 \rightarrow 9$ & \\
                                & $ 13x\;\ x = 4 \rightarrow 9$ &  	\\
                                & $ \Downarrow $ 	          &  	\\
                                & $ 18x\;\ x = 9 $ 	          & $ 7 + 6 + \cdots + 1 = 28 $ 	\\
                \end{tabular}\\$\;$\\\pause
                5+36+28=69 in total. \qed
        \end{frame}
    
        \begin{frame}[c]{\subsecname.4}
            For integers from 5 to 200, inclusive. How many are of the form xyz, where 0 $\neq$ x $<$ y and y $>$ z.\\$\;$\\\pause
            \centering
                \begin{tabular}{cr}
                     $ 12z\;\ z = 0,1$               & 2 \\
                     $ 13z\;\ z = 0 \rightarrow 2$   & 3 \\
                     $ \Downarrow $                  &  \\
                     $ 19z\;\ z = 0 \rightarrow 8$   & 9 
                \end{tabular}\\$\;$\\\pause
                $2+3+\cdots+9=44$ in total. \qed
        \end{frame}



    \subsection{Problem 13}
    
        \begin{frame}[c]{\subsecname}
            How many terms are there in the expansion of (x + y)(a + b + c)(e + f + g)(h + i):\\$\;$\\\pause
            \[2\times3\times3\times2=24\] \qed
        \end{frame}



    \subsection{Problem 14}
    
        \begin{frame}[c]{\subsecname}
            How many reflexive, symmetric, and antisymmetric relations are there on an n-element set?\\$\;$\\\pause
            \begin{itemize}
            \item symmetric and antisymmetric relation: self loop or no relation
            \item reflexive relation: every element has a self loop.
            \item Only one relation that every element has a self-loop, and no others. \qed
            \end{itemize}
        \end{frame}
        
        
        
\section*{Q \& A}

    \begin{frame}<handout:0>[c]{\secname}
        \centerline{\Huge{Questions about the problems?}}
    \end{frame}
    
    
    
\end{document}



